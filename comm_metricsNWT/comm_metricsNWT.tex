\documentclass[]{article}
\usepackage{lmodern}
\usepackage{amssymb,amsmath}
\usepackage{ifxetex,ifluatex}
\usepackage{fixltx2e} % provides \textsubscript
\ifnum 0\ifxetex 1\fi\ifluatex 1\fi=0 % if pdftex
  \usepackage[T1]{fontenc}
  \usepackage[utf8]{inputenc}
\else % if luatex or xelatex
  \ifxetex
    \usepackage{mathspec}
  \else
    \usepackage{fontspec}
  \fi
  \defaultfontfeatures{Ligatures=TeX,Scale=MatchLowercase}
\fi
% use upquote if available, for straight quotes in verbatim environments
\IfFileExists{upquote.sty}{\usepackage{upquote}}{}
% use microtype if available
\IfFileExists{microtype.sty}{%
\usepackage{microtype}
\UseMicrotypeSet[protrusion]{basicmath} % disable protrusion for tt fonts
}{}
\usepackage[margin=1in]{geometry}
\usepackage{hyperref}
\hypersetup{unicode=true,
            pdftitle={comm\_metricsNWT},
            pdfborder={0 0 0},
            breaklinks=true}
\urlstyle{same}  % don't use monospace font for urls
\usepackage{color}
\usepackage{fancyvrb}
\newcommand{\VerbBar}{|}
\newcommand{\VERB}{\Verb[commandchars=\\\{\}]}
\DefineVerbatimEnvironment{Highlighting}{Verbatim}{commandchars=\\\{\}}
% Add ',fontsize=\small' for more characters per line
\usepackage{framed}
\definecolor{shadecolor}{RGB}{248,248,248}
\newenvironment{Shaded}{\begin{snugshade}}{\end{snugshade}}
\newcommand{\KeywordTok}[1]{\textcolor[rgb]{0.13,0.29,0.53}{\textbf{#1}}}
\newcommand{\DataTypeTok}[1]{\textcolor[rgb]{0.13,0.29,0.53}{#1}}
\newcommand{\DecValTok}[1]{\textcolor[rgb]{0.00,0.00,0.81}{#1}}
\newcommand{\BaseNTok}[1]{\textcolor[rgb]{0.00,0.00,0.81}{#1}}
\newcommand{\FloatTok}[1]{\textcolor[rgb]{0.00,0.00,0.81}{#1}}
\newcommand{\ConstantTok}[1]{\textcolor[rgb]{0.00,0.00,0.00}{#1}}
\newcommand{\CharTok}[1]{\textcolor[rgb]{0.31,0.60,0.02}{#1}}
\newcommand{\SpecialCharTok}[1]{\textcolor[rgb]{0.00,0.00,0.00}{#1}}
\newcommand{\StringTok}[1]{\textcolor[rgb]{0.31,0.60,0.02}{#1}}
\newcommand{\VerbatimStringTok}[1]{\textcolor[rgb]{0.31,0.60,0.02}{#1}}
\newcommand{\SpecialStringTok}[1]{\textcolor[rgb]{0.31,0.60,0.02}{#1}}
\newcommand{\ImportTok}[1]{#1}
\newcommand{\CommentTok}[1]{\textcolor[rgb]{0.56,0.35,0.01}{\textit{#1}}}
\newcommand{\DocumentationTok}[1]{\textcolor[rgb]{0.56,0.35,0.01}{\textbf{\textit{#1}}}}
\newcommand{\AnnotationTok}[1]{\textcolor[rgb]{0.56,0.35,0.01}{\textbf{\textit{#1}}}}
\newcommand{\CommentVarTok}[1]{\textcolor[rgb]{0.56,0.35,0.01}{\textbf{\textit{#1}}}}
\newcommand{\OtherTok}[1]{\textcolor[rgb]{0.56,0.35,0.01}{#1}}
\newcommand{\FunctionTok}[1]{\textcolor[rgb]{0.00,0.00,0.00}{#1}}
\newcommand{\VariableTok}[1]{\textcolor[rgb]{0.00,0.00,0.00}{#1}}
\newcommand{\ControlFlowTok}[1]{\textcolor[rgb]{0.13,0.29,0.53}{\textbf{#1}}}
\newcommand{\OperatorTok}[1]{\textcolor[rgb]{0.81,0.36,0.00}{\textbf{#1}}}
\newcommand{\BuiltInTok}[1]{#1}
\newcommand{\ExtensionTok}[1]{#1}
\newcommand{\PreprocessorTok}[1]{\textcolor[rgb]{0.56,0.35,0.01}{\textit{#1}}}
\newcommand{\AttributeTok}[1]{\textcolor[rgb]{0.77,0.63,0.00}{#1}}
\newcommand{\RegionMarkerTok}[1]{#1}
\newcommand{\InformationTok}[1]{\textcolor[rgb]{0.56,0.35,0.01}{\textbf{\textit{#1}}}}
\newcommand{\WarningTok}[1]{\textcolor[rgb]{0.56,0.35,0.01}{\textbf{\textit{#1}}}}
\newcommand{\AlertTok}[1]{\textcolor[rgb]{0.94,0.16,0.16}{#1}}
\newcommand{\ErrorTok}[1]{\textcolor[rgb]{0.64,0.00,0.00}{\textbf{#1}}}
\newcommand{\NormalTok}[1]{#1}
\usepackage{graphicx,grffile}
\makeatletter
\def\maxwidth{\ifdim\Gin@nat@width>\linewidth\linewidth\else\Gin@nat@width\fi}
\def\maxheight{\ifdim\Gin@nat@height>\textheight\textheight\else\Gin@nat@height\fi}
\makeatother
% Scale images if necessary, so that they will not overflow the page
% margins by default, and it is still possible to overwrite the defaults
% using explicit options in \includegraphics[width, height, ...]{}
\setkeys{Gin}{width=\maxwidth,height=\maxheight,keepaspectratio}
\IfFileExists{parskip.sty}{%
\usepackage{parskip}
}{% else
\setlength{\parindent}{0pt}
\setlength{\parskip}{6pt plus 2pt minus 1pt}
}
\setlength{\emergencystretch}{3em}  % prevent overfull lines
\providecommand{\tightlist}{%
  \setlength{\itemsep}{0pt}\setlength{\parskip}{0pt}}
\setcounter{secnumdepth}{0}
% Redefines (sub)paragraphs to behave more like sections
\ifx\paragraph\undefined\else
\let\oldparagraph\paragraph
\renewcommand{\paragraph}[1]{\oldparagraph{#1}\mbox{}}
\fi
\ifx\subparagraph\undefined\else
\let\oldsubparagraph\subparagraph
\renewcommand{\subparagraph}[1]{\oldsubparagraph{#1}\mbox{}}
\fi

%%% Use protect on footnotes to avoid problems with footnotes in titles
\let\rmarkdownfootnote\footnote%
\def\footnote{\protect\rmarkdownfootnote}

%%% Change title format to be more compact
\usepackage{titling}

% Create subtitle command for use in maketitle
\providecommand{\subtitle}[1]{
  \posttitle{
    \begin{center}\large#1\end{center}
    }
}

\setlength{\droptitle}{-2em}

  \title{comm\_metricsNWT}
    \pretitle{\vspace{\droptitle}\centering\huge}
  \posttitle{\par}
    \author{}
    \preauthor{}\postauthor{}
      \predate{\centering\large\emph}
  \postdate{\par}
    \date{01 March 2019}


\begin{document}
\maketitle

\section{Overview}\label{overview}

The \textbf{comm\_metricsNWT} module computes a serie of community
metrics based on the bird predictions for The Northwestern Territories
(NWT). The computed metrics are: * Expected Species Richness :
\[S: 1 - \exp^{\lambda}\] * Shannon- Wiener diversity Index :
\[ H^' = - \sum_{i=1}^{R} p_ilnp_i \] * Simpson diversity Index:
\[ 1-D = \sum p_i^2 \] * Rao's Quadratic Entropy:
\[ \sum_{i=1}^{S-1} \sum_{j=i+1}^{S} d_ijp_ip_i\]

Species traits were obtained from \# Usage

\begin{Shaded}
\begin{Highlighting}[]
\KeywordTok{library}\NormalTok{(}\StringTok{"SpaDES"}\NormalTok{)}
\end{Highlighting}
\end{Shaded}

\begin{verbatim}
## loading reproducible     0.2.7.9000
## loading quickPlot        0.1.6
## loading SpaDES.core      0.2.5
## loading SpaDES.tools     0.3.1.9000
## loading SpaDES.addins    0.1.2
\end{verbatim}

\begin{verbatim}
## Default paths for SpaDES directories set to:
##   cachePath:  
##   inputPath:  C:\Users\AARAS2\AppData\Local\Temp\RtmpoJJ9yB/SpaDES/inputs
##   modulePath: C:\Users\AARAS2\AppData\Local\Temp\RtmpoJJ9yB/SpaDES/modules
##   outputPath: C:\Users\AARAS2\AppData\Local\Temp\RtmpoJJ9yB/SpaDES/outputs
## These can be changed using 'setPaths()'. See '?setPaths'.
\end{verbatim}

\begin{Shaded}
\begin{Highlighting}[]
\KeywordTok{library}\NormalTok{(}\StringTok{"raster"}\NormalTok{)}
\end{Highlighting}
\end{Shaded}

\begin{verbatim}
## Loading required package: sp
\end{verbatim}

\begin{Shaded}
\begin{Highlighting}[]
\CommentTok{# Source functions in R folder}
\KeywordTok{invisible}\NormalTok{(}\KeywordTok{sapply}\NormalTok{(}\DataTypeTok{X =} \KeywordTok{list.files}\NormalTok{(}\KeywordTok{file.path}\NormalTok{(}\KeywordTok{getwd}\NormalTok{(), }\StringTok{"R"}\NormalTok{), }\DataTypeTok{full.names =} \OtherTok{TRUE}\NormalTok{), }\DataTypeTok{FUN =}\NormalTok{ source))}

\CommentTok{# Set a storage project folder}
\KeywordTok{setPaths}\NormalTok{(}\DataTypeTok{modulePath =} \KeywordTok{file.path}\NormalTok{(}\StringTok{"C:/Users/AARAS2/Downloads/NWT/modules"}\NormalTok{))}
\end{Highlighting}
\end{Shaded}

\begin{verbatim}
## Setting:
##   options(
##     spades.modulePath = 'C:/Users/AARAS2/Downloads/NWT/modules'
##   )
\end{verbatim}

\begin{verbatim}
## Paths set to:
##   options(
##     reproducible.cachePath = 'C:/Users/AARAS2/AppData/Local/Temp/RtmpgLS47z/reproducible/cache'
##     spades.inputPath = 'C:/Users/AARAS2/AppData/Local/Temp/RtmpoJJ9yB/SpaDES/inputs'
##     spades.outputPath = 'C:/Users/AARAS2/AppData/Local/Temp/RtmpoJJ9yB/SpaDES/outputs'
##     spades.modulePath = 'C:/Users/AARAS2/Downloads/NWT/modules'
##   )
\end{verbatim}

\begin{Shaded}
\begin{Highlighting}[]
\NormalTok{paths <-}\StringTok{ }\KeywordTok{getPaths}\NormalTok{() }\CommentTok{# shows where the 4 relevant paths are}


\NormalTok{times <-}\StringTok{ }\KeywordTok{list}\NormalTok{(}\DataTypeTok{start =} \DecValTok{0}\NormalTok{, }\DataTypeTok{end =} \DecValTok{3}\NormalTok{)}

\NormalTok{parameters <-}\StringTok{ }\KeywordTok{list}\NormalTok{(}
  \DataTypeTok{.progress =} \KeywordTok{list}\NormalTok{(}\DataTypeTok{type =} \StringTok{"text"}\NormalTok{, }\DataTypeTok{interval =} \DecValTok{1}\NormalTok{), }\CommentTok{# for a progress bar}
\NormalTok{  ## If there are further modules, each can have its own set of parameters:}
\DataTypeTok{birdsNWT =} \KeywordTok{list}\NormalTok{(}
    \StringTok{"baseLayer"}\NormalTok{ =}\StringTok{ }\DecValTok{2005}\NormalTok{,}
    \StringTok{"overwritePredictions"}\NormalTok{ =}\StringTok{ }\OtherTok{TRUE}\NormalTok{,}
    \StringTok{"useTestSpeciesLayers"}\NormalTok{ =}\StringTok{ }\OtherTok{TRUE}\NormalTok{, }\CommentTok{# Set it to false when you actually have results from LandR_Biomass simulations to run it with}
    \StringTok{"useParallel"}\NormalTok{ =}\StringTok{ }\OtherTok{FALSE}\NormalTok{, }\CommentTok{# Using parallel in windows is currently not working.}
    \StringTok{"predictionInterval"}\NormalTok{ =}\StringTok{ }\DecValTok{1}
\NormalTok{  )}
\NormalTok{)}
\NormalTok{.objects <-}\StringTok{ }\KeywordTok{list}\NormalTok{(}
  \StringTok{"birdsList"}\NormalTok{ =}\StringTok{ }\KeywordTok{c}\NormalTok{(}\StringTok{"BBWA"}\NormalTok{, }\StringTok{"BOCH"}\NormalTok{,}\StringTok{"RBNU"}\NormalTok{))}
\NormalTok{modules <-}\StringTok{ }\KeywordTok{list}\NormalTok{(}\StringTok{"birdsNWT"}\NormalTok{, }\StringTok{"comm_metricsNWT"}\NormalTok{)}
\NormalTok{inputs <-}\StringTok{ }\KeywordTok{list}\NormalTok{()}
\NormalTok{outputs <-}\StringTok{ }\KeywordTok{list}\NormalTok{()}

\NormalTok{comm_metricsNWT <-}\StringTok{ }\KeywordTok{simInitAndSpades}\NormalTok{(}\DataTypeTok{times =}\NormalTok{ times, }\DataTypeTok{params =}\NormalTok{ parameters, }\DataTypeTok{modules =}\NormalTok{ modules,}
                 \DataTypeTok{objects =}\NormalTok{ .objects, }\DataTypeTok{loadOrder =} \KeywordTok{c}\NormalTok{(}\StringTok{"birdsNWT"}\NormalTok{,}\StringTok{"comm_metricsNWT"}\NormalTok{), }\DataTypeTok{debug =} \DecValTok{2}\NormalTok{)}
\end{Highlighting}
\end{Shaded}

\begin{verbatim}
## Setting:
##   options(
##     reproducible.cachePath = 'C:/Users/AARAS2/AppData/Local/Temp/RtmpgLS47z/reproducible/cache'
##     spades.inputPath = 'C:/Users/AARAS2/AppData/Local/Temp/RtmpoJJ9yB/SpaDES/inputs'
##     spades.outputPath = 'C:/Users/AARAS2/AppData/Local/Temp/RtmpoJJ9yB/SpaDES/outputs'
##     spades.modulePath = 'C:/Users/AARAS2/Downloads/NWT/modules'
##   )
\end{verbatim}

\begin{verbatim}
## Loading required package: crayon
\end{verbatim}

\begin{verbatim}
## Loading required package: data.table
\end{verbatim}

\begin{verbatim}
## 
## Attaching package: 'data.table'
\end{verbatim}

\begin{verbatim}
## The following object is masked from 'package:raster':
## 
##     shift
\end{verbatim}

\begin{verbatim}
## Loading required package: dplyr
\end{verbatim}

\begin{verbatim}
## 
## Attaching package: 'dplyr'
\end{verbatim}

\begin{verbatim}
## The following objects are masked from 'package:data.table':
## 
##     between, first, last
\end{verbatim}

\begin{verbatim}
## The following objects are masked from 'package:raster':
## 
##     intersect, select, union
\end{verbatim}

\begin{verbatim}
## The following objects are masked from 'package:stats':
## 
##     filter, lag
\end{verbatim}

\begin{verbatim}
## The following objects are masked from 'package:base':
## 
##     intersect, setdiff, setequal, union
\end{verbatim}

\begin{verbatim}
## Loading required package: gbm
\end{verbatim}

\begin{verbatim}
## Loaded gbm 2.1.5
\end{verbatim}

\begin{verbatim}
## Loading required package: googledrive
\end{verbatim}

\begin{verbatim}
## Loading required package: plyr
\end{verbatim}

\begin{verbatim}
## -------------------------------------------------------------------------
\end{verbatim}

\begin{verbatim}
## You have loaded plyr after dplyr - this is likely to cause problems.
## If you need functions from both plyr and dplyr, please load plyr first, then dplyr:
## library(plyr); library(dplyr)
\end{verbatim}

\begin{verbatim}
## -------------------------------------------------------------------------
\end{verbatim}

\begin{verbatim}
## 
## Attaching package: 'plyr'
\end{verbatim}

\begin{verbatim}
## The following objects are masked from 'package:dplyr':
## 
##     arrange, count, desc, failwith, id, mutate, rename, summarise,
##     summarize
\end{verbatim}

\begin{verbatim}
## defineParameter: 'nCores' is not of specified type 'character|numeric'.
\end{verbatim}

\begin{verbatim}
## defineParameter: 'baseLayer' is not of specified type 'character'.
\end{verbatim}

\begin{verbatim}
## birdsNWT: module code: birdPrediction is declared in metadata outputObjects, but is not assigned in the module
\end{verbatim}

\begin{verbatim}
## birdsNWT: module code: wetlandRaster, cloudFolderID, urlModels, urlStaticLayers are declared in metadata inputObjects, but no default(s) are provided in .inputObjects
\end{verbatim}

\begin{verbatim}
## birdsNWT: module code: wetlandRaster, cloudFolderID, urlModels, urlStaticLayers are declared in metadata inputObjects, but are not used in the module
\end{verbatim}

\begin{verbatim}
## birdsNWT: module code: classifyWetlands: parameter 'rasterToMatch' may not be used
\end{verbatim}

\begin{verbatim}
## birdsNWT: module code: corePrediction : <anonymous>: parameter 'e' may not be used
\end{verbatim}

\begin{verbatim}
## birdsNWT: module code: createSpeciesStackLayer : <anonymous>: local variable 'speciesLayerNames' assigned but may not be used
\end{verbatim}

\begin{verbatim}
## birdsNWT: module code: createSpeciesStackLayer: parameter 'pathData' may not be used
\end{verbatim}

\begin{verbatim}
## birdsNWT: module code: loadBirdModels: parameter 'cloudFolderID' may not be used
\end{verbatim}

\begin{verbatim}
## birdsNWT: module code: loadTestSpeciesLayers: parameter 'modelList' may not be used
\end{verbatim}

\begin{verbatim}
## birdsNWT: module code: loadTestSpeciesLayers: parameter 'successionTables' may not be used
\end{verbatim}

\begin{verbatim}
## birdsNWT: outputObjects: Year0 is assigned to sim inside doEvent.birdsNWT, but is not declared in metadata outputObjects
\end{verbatim}

\begin{verbatim}
## birdsNWT: inputObjects: simulatedBiomassMap, cohortData, pixelGroupMap, sppEquiv, Year0 are used from sim inside doEvent.birdsNWT, but are not declared in metadata inputObjects
\end{verbatim}

\begin{verbatim}
## Warning in FUN(X[[i]], ...): is.na() applied to non-(list or vector) of
## type 'closure'
\end{verbatim}

\begin{verbatim}
## comm_metricsNWT: module code: birdPrediction, birdsList are declared in metadata inputObjects, but no default(s) are provided in .inputObjects
\end{verbatim}

\begin{verbatim}
## comm_metricsNWT: module code: birdsList is declared in metadata inputObjects, but is not used in the module
\end{verbatim}

\begin{verbatim}
## comm_metricsNWT: outputObjects: 0 is assigned to sim inside calcDiversity, but is not declared in metadata outputObjects
\end{verbatim}

\begin{verbatim}
## comm_metricsNWT: outputObjects: currentDiversityRasters is assigned to sim inside calcDiversityIndices, but is not declared in metadata outputObjects
\end{verbatim}

\begin{verbatim}
## comm_metricsNWT: inputObjects: birdList is assigned to sim inside .inputObjects, but is not declared in metadata inputObjects
\end{verbatim}

\begin{verbatim}
## comm_metricsNWT: inputObjects: 0, currentDiversityRasters are used from sim inside calcDiversity, but are not declared in metadata inputObjects
\end{verbatim}

\begin{verbatim}
## comm_metricsNWT: inputObjects: currentDiversityRasters is used from sim inside diversityPlot, but is not declared in metadata inputObjects
\end{verbatim}

\begin{verbatim}
## comm_metricsNWT: inputObjects: currentDiversityRasters is used from sim inside diversityStats, but is not declared in metadata inputObjects
\end{verbatim}

\begin{verbatim}
## comm_metricsNWT: inputObjects: currentDiversityRasters is used from sim inside doEvent.comm_metricsNWT, but is not declared in metadata inputObjects
\end{verbatim}

\begin{verbatim}
## Running .inputObjects for birdsNWT
\end{verbatim}

\begin{verbatim}
## birdsNWT: using dataPath 'C:/Users/AARAS2/Downloads/NWT/modules/birdsNWT/data'.
\end{verbatim}

\begin{verbatim}
## No specific study area was provided. Croping to the Edehzhie Indigenous Protected Area (Southern NWT)
\end{verbatim}

\begin{verbatim}
##   loading cached result from previous prepInputs call, adding to memoised copy
##   loading cached result from previous prepInputs call, adding to memoised copy
\end{verbatim}

\begin{verbatim}
##   loading cached result from previous prepInputsLayers_DUCKS call, adding to memoised copy
\end{verbatim}

\begin{verbatim}
##   loading cached result from previous classifyWetlands call, adding to memoised copy
\end{verbatim}

\begin{verbatim}
##   loading cached result from previous cropInputs call, adding to memoised copy
\end{verbatim}

\begin{verbatim}
##   loading cached result from previous projectInputs call, adding to memoised copy
\end{verbatim}

\begin{verbatim}
##   loading cached result from previous maskInputs call, adding to memoised copy
\end{verbatim}

\begin{verbatim}
## Running .inputObjects for comm_metricsNWT
\end{verbatim}

\begin{verbatim}
## [1] "13:00:33 | elpsd: 0.001 secs | 0 checkpoint init 5"
## [1] "13:00:33 | elpsd: 0.004 secs | 0 save init 5"
## [1] "13:00:33 | elpsd: 0.0029 secs | 0 progress init 5"
## [1] "13:00:33 | elpsd: 0.001 secs | 0 load init 5"
## [1] "13:00:33 | elpsd: 0.001 secs | 0 birdsNWT init 5"
## [1] "13:00:33 | elpsd: 0.001 secs | 0 comm_metricsNWT init 5"
## [1] "13:00:33 | elpsd: 0.001 secs | 0 birdsNWT loadModels 5"
\end{verbatim}

\begin{verbatim}
##   loading cached result from previous loadBirdModels call, adding to memoised copy
\end{verbatim}

\begin{verbatim}
## Bird models loaded for: 
## BBWA
## BOCH
## RBNU
\end{verbatim}

\begin{verbatim}
## [1] "13:00:42 | elpsd: 9.4 secs | 0 birdsNWT loadFixedLayers 5"
\end{verbatim}

\begin{verbatim}
##   loading cached result from previous loadStaticLayers call, adding to memoised copy
\end{verbatim}

\begin{verbatim}
## The following static layers have been loaded: 
## wat
## led25
## urbag
## dev25
## landform
\end{verbatim}

\begin{verbatim}
## [1] "13:00:43 | elpsd: 0.87 secs | 0 birdsNWT gettingData 5"
\end{verbatim}

\begin{verbatim}
## Loading required package: magrittr
\end{verbatim}

\begin{verbatim}
## 
## Attaching package: 'magrittr'
\end{verbatim}

\begin{verbatim}
## The following object is masked from 'package:raster':
## 
##     extract
\end{verbatim}

\begin{verbatim}
## cohortData not supplied by another module. Will try using files in inputPath(sim)
\end{verbatim}

\begin{verbatim}
## pixelGroupMap not supplied by another module. Will try using files in inputPath(sim)
\end{verbatim}

\begin{verbatim}
## simulatedBiomassMap not supplied by another module. Will try using files in inputPath(sim)
\end{verbatim}

\begin{verbatim}
## [1] "13:00:43 | elpsd: 0.086 secs | 0 birdsNWT predictBirds 10"
\end{verbatim}

\begin{verbatim}
## Using test layers for species. Predictions will be static and identical to original data.
\end{verbatim}

\begin{verbatim}
##   loading cached result from previous loadTestSpeciesLayers call, adding to memoised copy
\end{verbatim}

\begin{verbatim}
##   loading cached result from previous predictDensities call, adding to memoised copy
\end{verbatim}

\begin{verbatim}
## [1] "13:00:57 | elpsd: 14 secs | 0 comm_metricsNWT calcDiversity 10"
## [1] "13:01:06 | elpsd: 9.4 secs | 0 comm_metricsNWT diversityStats 10"
## [1] "13:01:07 | elpsd: 0.45 secs | 1 birdsNWT gettingData 5"
\end{verbatim}

\begin{verbatim}
## cohortData not supplied by another module. Will try using files in inputPath(sim)
\end{verbatim}

\begin{verbatim}
## pixelGroupMap not supplied by another module. Will try using files in inputPath(sim)
\end{verbatim}

\begin{verbatim}
## simulatedBiomassMap not supplied by another module. Will try using files in inputPath(sim)
\end{verbatim}

\begin{verbatim}
## [1] "13:01:07 | elpsd: 0.05 secs | 1 birdsNWT predictBirds 5"
\end{verbatim}

\begin{verbatim}
## Using test layers for species. Predictions will be static and identical to original data.
\end{verbatim}

\begin{verbatim}
##   loading memoised result from previous loadTestSpeciesLayers call.
\end{verbatim}

\begin{verbatim}
##   loading cached result from previous predictDensities call, adding to memoised copy
\end{verbatim}

\begin{verbatim}
## [1] "13:01:20 | elpsd: 13 secs | 1 comm_metricsNWT plot 10"
## [1] "13:01:20 | elpsd: 0.46 secs | 1 comm_metricsNWT calcDiversity 10"
## [1] "13:01:29 | elpsd: 8.7 secs | 1 comm_metricsNWT diversityStats 10"
## [1] "13:01:29 | elpsd: 0.24 secs | 2 birdsNWT gettingData 5"
\end{verbatim}

\begin{verbatim}
## cohortData not supplied by another module. Will try using files in inputPath(sim)
\end{verbatim}

\begin{verbatim}
## pixelGroupMap not supplied by another module. Will try using files in inputPath(sim)
\end{verbatim}

\begin{verbatim}
## simulatedBiomassMap not supplied by another module. Will try using files in inputPath(sim)
\end{verbatim}

\begin{verbatim}
## [1] "13:01:29 | elpsd: 0.035 secs | 2 birdsNWT predictBirds 5"
\end{verbatim}

\begin{verbatim}
## Using test layers for species. Predictions will be static and identical to original data.
\end{verbatim}

\begin{verbatim}
##   loading memoised result from previous loadTestSpeciesLayers call.
\end{verbatim}

\begin{verbatim}
##   loading cached result from previous predictDensities call, adding to memoised copy
\end{verbatim}

\begin{verbatim}
## [1] "13:01:41 | elpsd: 12 secs | 2 comm_metricsNWT plot 10"
## [1] "13:01:42 | elpsd: 0.44 secs | 2 comm_metricsNWT calcDiversity 10"
## [1] "13:01:51 | elpsd: 8.8 secs | 2 comm_metricsNWT diversityStats 10"
## [1] "13:01:51 | elpsd: 0.23 secs | 3 birdsNWT gettingData 5"
\end{verbatim}

\begin{verbatim}
## cohortData not supplied by another module. Will try using files in inputPath(sim)
\end{verbatim}

\begin{verbatim}
## pixelGroupMap not supplied by another module. Will try using files in inputPath(sim)
\end{verbatim}

\begin{verbatim}
## simulatedBiomassMap not supplied by another module. Will try using files in inputPath(sim)
\end{verbatim}

\begin{verbatim}
## [1] "13:01:51 | elpsd: 0.066 secs | 3 birdsNWT predictBirds 5"
\end{verbatim}

\begin{verbatim}
## Using test layers for species. Predictions will be static and identical to original data.
\end{verbatim}

\begin{verbatim}
##   loading memoised result from previous loadTestSpeciesLayers call.
\end{verbatim}

\begin{verbatim}
##   loading cached result from previous predictDensities call, adding to memoised copy
\end{verbatim}

\includegraphics{comm_metricsNWT_files/figure-latex/module_usage-1.pdf}

\begin{verbatim}
## [1] "13:02:05 | elpsd: 14 secs | 3 comm_metricsNWT plot 10"
## [1] "13:02:06 | elpsd: 0.23 secs | 3 comm_metricsNWT calcDiversity 10"
## [1] "13:02:15 | elpsd: 9.7 secs | 3 comm_metricsNWT diversityStats 10"
\end{verbatim}

\section{Events}\label{events}

Describe what happens for each event type.

\subsection{Plotting}\label{plotting}

Write what is plotted.

\subsection{Saving}\label{saving}

Write what is saved.

\section{Data dependencies}\label{data-dependencies}

\subsection{Input data}\label{input-data}

How to obtain input data, and a description of the data required by the
module. If \texttt{sourceURL} is specified,
\texttt{downloadData("comm\_metricsNWT",\ "path/to/modules/dir")} may be
sufficient.

\subsection{Output data}\label{output-data}

Description of the module outputs.

\section{Links to other modules}\label{links-to-other-modules}

Describe any anticipated linkages to other modules.


\end{document}
